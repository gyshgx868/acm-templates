\subsection{排列组合的计算}

排列的定义:

\begin{equation}
  A^m_n = n(n-1)(n-2) \cdots (n-m+1)
\end{equation}

组合的定义:

\begin{equation}
  C^m_n = \frac{A^m_n}{m!} = \frac{n(n-1)(n-2) \cdots (n-m+1)}{m!}
\end{equation}

组合数的计算:
\begin{itemize}
  \item 加法:$C^m_n=C^{m-1}_{n-1}+C^m_{n-1}$,边界条件为$C^0_n=C^n_n=1$
  \item 乘法(注意不要改变乘除顺序):
    \begin{equation}
      C^m_n = \left \lbrace
        \begin{array}{ll}
          1	&	(m=0)	\\
          \frac{(n-m+1)C^{m-1}_n}{m} & (0<m \leq \frac{n}{2}) \\
          C^{n-m}_n & (\frac{n}{2}<m \leq n)
        \end{array}
      \right.
    \end{equation}
\end{itemize}

注意:很多排列组合问题的答案确实可以用数学表达式来表示,但是,数学表达式可能并不适合计算机,因此,\textbf{尝试寻找递推关系往往比直接找数学关系更好。}

\subsection{排列组合的生成}

\subsubsection{类循环排列}

示例:0000、0001、0002、0010、0011、0012、0020、0021、0022、0100、0101……2222

在数字位数和可用数字个数不确定的情况下,显然无法用多重循环——用递归就可以了。

注意,某些问题也可以用位运算来解决。例如只有0和1的时候,或者数字是0、1、2、3的时候。

\subsubsection{全排列 (STL)}

\begin{itemize}
  \item next\_permutation()函数将按字母表顺序生成给定序列的下一个较大的排列,如果还有下一个排列,那么函数返回true,否则返回false;
  \item prev\_permutation()函数与之相反,是生成给定序列的上一个较小的排列。两个函数都在<algorithm>头文件里面。
\end{itemize}

\begin{lstlisting}
int main() {
  int a[] = {1, 2, 3};
  do {
    for (int i = 0; i < 3; i++) { printf("%d ", a[i]); }
    printf("\n");
  } while (next_permutation(a, a + 3));
  return 0;
}
\end{lstlisting}

\subsubsection{一般组合}

从$n$个元素中任取$m$个元素,请输出所有取法。假设有1到5这5个元素,从中取3个,则序列应该为:123、124、125、134、135、145、234、235、245、345。

\lstinputlisting{math/combinatorics/combination.cpp}

\subsubsection{求全部子集}

以下代码求出的集合包含空集和原集合。

\lstinputlisting{math/combinatorics/subsets.cpp}

\subsubsection{由上一排列生成下一排列}

\paragraph{示例} 153642经过变换之后应该是154236。

\paragraph{思路} 从右向左扫描,找到第一个破坏顺序的数$p$(原来从右向左时数字是从小到大变化的),如果找不到,说明此排列已经是最后一个排列了。接下来在$p$的右边找到大于$p$的第一个数,并与$p$交换。之后将原来$p$位置右面的所有数倒序即可。

\lstinputlisting{math/combinatorics/next_permutation.cpp}

\subsection{组合数}

\subsubsection{暴力求组合数}

\paragraph{注} 需要快速幂算法quick\_pow,以及快速乘法quick\_mul。

\lstinputlisting{math/combinatorics/brute_combination.cpp}

\subsubsection{组合数打表}
\lstinputlisting{math/combinatorics/combination_table.cpp}

\subsubsection{组合数在线}

\paragraph{注} 需要求解逆元函数get\_inv,调用函数之前,需要使用init初始化。

\lstinputlisting{math/combinatorics/get_combination.cpp}

\subsection{Lucas定理}

\paragraph{注} 需要组合数在线算法。

\lstinputlisting{math/combinatorics/lucas.cpp}

\subsection{贝尔数}

贝尔数给出了集合划分的数目,集合S的一个划分是定义为S的两两不相交的非空子集的族,它们的并是S。

3个元素的集合$\{a,b,c\}$有5种不同的划分方法:$\{\{a\},\{b\},\{c\}\}$、$\{\{a\},\{b,c\}\}$、$\{\{b\},\{a,c\}\}$、$\{\{c\},\{a,b\}\}$、$\{\{a,b,c\}\}$。

划分的定义要求空集的划分中的每个成员都是非空集合。

\paragraph{注} 需要组合数打表。

\lstinputlisting{math/combinatorics/get_bells.cpp}

\subsection{Gray码}

格雷码是一个数列集合,每个数使用二进位来表示,假设使用$n$位元来表示每个数字,任两个数之间只有一个位元值不同。

例如以下为3位元的格雷码:000,001,011,010,110,111,101,100。

\lstinputlisting{math/combinatorics/gray.cpp}

\subsection{幻方构造}

幻方(Magic Square)是一种将数字安排在正方形格子中,使每行、列和对角线上的数字和都相等的方法。

\lstinputlisting{math/combinatorics/magic_square.cpp}
