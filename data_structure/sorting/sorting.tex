\subsection{快速排序}

快速排序的时间复杂度为$\mathcal{O}(n\log n)$,但是极端情况(数据基本有序)下会退化成$\mathcal{O}(n^2)$。因此建议使用STL的sort()函数。STL的sort()与上面代码相比,具有以下特点:

\begin{itemize}
  \item 数据量大时采用分段递归排序,即快速排序。在取分隔点时,取的是头部、尾部和中央三个元素的中间值;
  \item 数据量变小的时候,采用插入排序代替快速排序;
  \item 此外,快速排序是不稳定的排序算法;
  \item 如果递归层次过深,会改用堆排序。
\end{itemize}

\lstinputlisting{data_structure/sorting/quick_sort.cpp}

\subsection{归并排序}

归并排序的时间复杂度为$\mathcal{O}(n\log n)$,但是空间复杂度很大,为$\mathcal{O}(n)$。归并排序是稳定的排序算法,即数值相同时,元素的相对位置不会发生改变。STL的stable\_sort()采用了归并排序算法。

\lstinputlisting{data_structure/sorting/merge_sort.cpp}

\subsection{堆排序}

堆排序的时间复杂度为$\mathcal{O}(n\log n)$。但是由于该算法常数因子有些大,因此它比快速排序慢很多。不过它不需要递归,所以不怕爆栈。堆排序的思路:

\begin{enumerate}
  \item 将整个数组转化为一个堆。如果想把一串数从小到大排序,则需要使用最大值堆1;
  \item 将堆顶的最大元素取出,并把它放到数组的最后;
  \item 剩余元素重新建堆;
  \item 重复第2步,直到堆为空。
\end{enumerate}

\lstinputlisting{data_structure/sorting/heap_sort.cpp}
