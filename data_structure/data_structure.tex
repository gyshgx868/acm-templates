\section{排序}
\subsection{快速排序}

快速排序的时间复杂度为$\mathcal{O}(n\log n)$,但是极端情况(数据基本有序)下会退化成$\mathcal{O}(n^2)$。因此建议使用STL的sort()函数。STL的sort()与上面代码相比,具有以下特点:

\begin{itemize}
  \item 数据量大时采用分段递归排序,即快速排序。在取分隔点时,取的是头部、尾部和中央三个元素的中间值;
  \item 数据量变小的时候,采用插入排序代替快速排序;
  \item 此外,快速排序是不稳定的排序算法;
  \item 如果递归层次过深,会改用堆排序。
\end{itemize}

\lstinputlisting{data_structure/sorting/quick_sort.cpp}

\subsection{归并排序}

归并排序的时间复杂度为$\mathcal{O}(n\log n)$,但是空间复杂度很大,为$\mathcal{O}(n)$。归并排序是稳定的排序算法,即数值相同时,元素的相对位置不会发生改变。STL的stable\_sort()采用了归并排序算法。

\lstinputlisting{data_structure/sorting/merge_sort.cpp}

\subsection{堆排序}

堆排序的时间复杂度为$\mathcal{O}(n\log n)$。但是由于该算法常数因子有些大,因此它比快速排序慢很多。不过它不需要递归,所以不怕爆栈。堆排序的思路:

\begin{enumerate}
  \item 将整个数组转化为一个堆。如果想把一串数从小到大排序,则需要使用最大值堆1;
  \item 将堆顶的最大元素取出,并把它放到数组的最后;
  \item 剩余元素重新建堆;
  \item 重复第2步,直到堆为空。
\end{enumerate}

\lstinputlisting{data_structure/sorting/heap_sort.cpp}


\section{树状数组}
\subsection{一维树状数组}

\subsubsection{普通版本}
\lstinputlisting{data_structure/fenwick_tree/fenwick_tree.cpp}

\subsubsection{区间更新单点查询}
\lstinputlisting{data_structure/fenwick_tree/fenwick_tree_range_update_point_query.cpp}

\subsubsection{区间更新区间查询}

\paragraph{例题} POJ3468

\lstinputlisting{data_structure/fenwick_tree/fenwick_tree_range_update_range_query.cpp}

\subsection{二维树状数组}
\lstinputlisting{data_structure/fenwick_tree/fenwick_tree_2d.cpp}

\subsection{三维树状数组}
\lstinputlisting{data_structure/fenwick_tree/fenwick_tree_3d.cpp}


\section{逆序对}
对于一个序列$a_1, a_2, a_3, ... , a_n$,如果存在$i<j$,使$a_i>a_j$,那么$(a_i,a_j)$就是一个逆序对。

\subsection{逆序对 (归并排序)}
\lstinputlisting{data_structure/inversion/inversion_merge_sort.cpp}

\subsection{逆序对 (树状数组)}

由于我们只是看两个数之间的大小关系,所以可以对序列中的数进行离散化。即按照大小关系把$a_1$到$a_n$映射到1至num之间(num为不同数字的个数),保证仍然满足原有的大小关系。这样,本题就转化成了:对于一个数$a_i$,在它后面有多少个比它小的数。

\lstinputlisting{data_structure/inversion/inversion_fenwick_tree.cpp}


\section{并查集}
\lstinputlisting{data_structure/disjoint_set.cpp}

\section{线段树}
\subsection{线段树}
\lstinputlisting{data_structure/interval_tree/interval_tree.cpp}

\subsection{区间求和}

使用下面的模板类特化IntervalTree中的Op:IntervalTree<int, SumOperation<int> >。

\paragraph{例题} HDU1166(单点更新区间求和)

\lstinputlisting{data_structure/interval_tree/sum_interval_tree.cpp}

\subsection{区间求最值}

\paragraph{例题} HDU1754(单点更新区间求最大值)

\paragraph{注} 下面的模板为区间求最大值,若需要更改为区间最小值,则需要将max函数改为min,并将最小值的初始化改为最大值。

\lstinputlisting{data_structure/interval_tree/extremum_interval_tree.cpp}

\subsection{二维线段树单点更新区间求和}

\paragraph{例题} POJ1195

\lstinputlisting{data_structure/interval_tree/interval_tree_2d.cpp}


\section{可持久化数据结构}
\subsection{可持久化线段树}

求解区间第$k$小,下标从$0$开始。

\lstinputlisting{data_structure/persistency/persistent_tree.cpp}


\section{RMQ}
RMQ(Range Minimum/Maximum Query)问题是指:已知长度为$n$的数列$A$,需要查询某个区间的最值若干次。

由于询问次数很多,并且区间范围不确定,因此采用朴素算法是不可行的。若采用线段树,预处理的时间为$\mathcal{O}(n)$,查询的时间是$\mathcal{O}(\log n)$。并且,如果规定数列元素可以更改,那么基本上只能用线段树来处理。这种情况下,该算法不可用。

RMQ是在线算法,预处理的时间为$\mathcal{O}(n\log n)$,但回答一次询问的仅为$\mathcal{O}(1)$。

\subsection{RMQ}

\lstinputlisting{data_structure/sparse_table/sparse_table.cpp}

\paragraph{注} 可以将上述代码中的Min替换为Max或者gcd来维护区间最小值或者最大公约数,仿函数的写法如下:

\lstinputlisting{data_structure/sparse_table/operation.cpp}

\subsection{RMQ 维护区间和}
\lstinputlisting{data_structure/sparse_table/sum_sparse_table.cpp}

\subsection{2D RMQ}

\paragraph{例题} HDU2888(该题需要将vector替换为数组,否则会MLE)

\lstinputlisting{data_structure/sparse_table/sparse_table_2d.cpp}


\section{最近公共祖先 (LCA)}
求LCA的Tarjan算法是一个经典的离线算法。

Tarjan算法用到了并查集。LCA问题可以用$\mathcal{O}(n+Q)$的时间来解决,其中$Q$为询问的次数。

Tarjan算法基于深度优先搜索的框架。对于新搜索到的一个结点,首先创建由这个结点构成的集合,再对当前结点的每一个子树进行搜索,每搜索完一棵子树,则可确定子树内的LCA询问都已解决,其他的LCA询问的结果必然在这个子树之外。

这时把子树所形成的集合与当前结点的集合合并,并将当前结点设为这个集合的祖先。之后继续搜索下一棵子树,直到当前结点的所有子树搜索完。这时把当前结点也设为“已被检查过的”,同时可以处理有关当前结点的LCA询问,如果有一个从当前结点到结点$v$的询问,且$v$已被检查过,那么,由于进行的是深度优先搜索,所以当前结点与$v$的最近公共祖先一定还没有被检查,而这个最近公共祖先的包含$v$的子树一定已经搜索过了,因此这个最近公共祖先一定是$v$所在集合的祖先。

\subsection{Tarjan离线算法}

\lstinputlisting{data_structure/lca/tarjan_offline.cpp}

\subsection{RMQ在线算法}

预处理的复杂度为$\mathcal{O}(n\log n)$,查询复杂度为$\mathcal{O}(1)$:\href{https://www.geeksforgeeks.org/lca-n-ary-tree-constant-query-o1}{https://www.geeksforgeeks.org/lca-n-ary-tree-constant-query-o1}。

\lstinputlisting{data_structure/lca/lca_rmq.cpp}

\subsection{倍增算法}

预处理的复杂度为$\mathcal{O}(n\log n)$,查询复杂度为$\mathcal{O}(\log n)$:\href{https://www.geeksforgeeks.org/lca-for-general-or-n-ary-trees-sparse-matrix-dp-approach-onlogn-ologn}{https://www.geeksforgeeks.org/lca-for-general-or-n-ary-trees-sparse-matrix-dp-approach-onlogn-ologn}。

\lstinputlisting{data_structure/lca/lca_sparse_matrix.cpp}


\section{树链剖分}
\subsection{基于点权}

查询单点值,修改路径上的点权(HDU 3966,树链剖分 + 树状数组)。

\lstinputlisting{data_structure/heavy_light_decomposition/vertex_weight.cpp}

\subsection{基于边权}

修改单条边权,查询边权的最大值(SPOJ QTREE,树链剖分 + 线段树)。

\lstinputlisting{data_structure/heavy_light_decomposition/edge_weight.cpp}


\section{二叉排序树}
\subsection{Size Balanced Tree (SBT)}
\lstinputlisting{data_structure/balanced_tree/sbt.cpp}

\subsection{伸展树 (Splay Tree)}
\lstinputlisting{data_structure/balanced_tree/splay.cpp}

\subsection{C++的set和map}

map(映射)、multimap(多重映射)、set(集合)、multiset(多重集合)属于关联容器,头文件分别位于<map>和<set>中。

map和set的区别:set实际上就是一组元素的集合,但其中所包含的元素的值是唯一的,且是按一定顺序排列的。集合中的每个元素被称作集合中的实例。其内部通过链表的方式来组织;而map提供一种“键—值”关系的一对一的数据存储能力,类似于字典。其“键”在容器中不可重复,且按一定顺序排列。由于其是按链表的方式存储,它也继承了链表的优缺点。

multiset和set不同之处在于,multiset中元素的值可以不唯一。multimap也类似,在multimap中“键”可以不唯一。

关联容器的特点:

\begin{enumerate}
  \item 关联容器对元素的插入和删除操作比vector快,但比list慢;
  \item 关联容器对元素的检索操作比vector慢,但是比list要快很多。关联容器查找的复杂度基本是$\mathcal{O}(\log n)$;
  \item set是内部排序的,这与序列容器有着本质的区别。
\end{enumerate}

用法详见第三章STL。
    
\subsection{Treap}
\lstinputlisting{data_structure/balanced_tree/treap.cpp}

\subsection{替罪羊树}

替罪羊树是计算机科学中,一种基于部分重建的自平衡二叉搜索树。在替罪羊树上,插入或删除节点的平摊最坏时间复杂度是$\mathcal{O}(\log n)$,搜索节点的最坏时间复杂度是$\mathcal{O}(\log n)$。

\lstinputlisting{data_structure/balanced_tree/scapegoat_tree.cpp}


\section{二叉堆}
\subsection{基本二叉堆}

二叉堆:二叉堆是完全二叉树。按照树根大小,二叉堆可分为最大值堆和最小值堆。

二叉堆的特点:

\begin{enumerate}
  \item 最大(小)值堆中,结点一定不小(大)于两个儿子的值;
  \item 在堆中,两兄弟的大小没有必然联系;
  \item 最大(小)值堆的根结点是整个树中的最大(小)值。
\end{enumerate}

由于是完全二叉树,所以可以直接用一维数组保存。数组的下标是从0开始的。

二叉堆的操作有:

\begin{enumerate}
  \item 插入:在堆中插入元素,首先要把元素放到末尾,然后通过不断往上“拱”,把元素“拱”到正确的位置;
  \item 用现有值初始化:最快的方法不是挨个插入,而是直接调整数组元素的顺序,使其符合堆的性质;
  \item 查找:查找最值是最快的——直接访问树根就可以了。不过,用堆查找其他值就很慢了。因此,可以考虑再使用一个适合查找的辅助数据结构,例如二叉排序树;
  \item 删除:把堆中最后一个元素 (就是一维数组存储所对应的最后一个元素) 放到待删除元素的位置,将元素总数减1,然后调整各元素的顺序。
\end{enumerate}

\lstinputlisting{data_structure/binary_heap/heap.cpp}

\subsection{STL中的堆算法}

为了实现堆算法,需要一个支持随机迭代器的容器。当然,一维数组也可以。

下面各函数的comp用于代替默认的小于号。如果不需要,可以省略。如果不指明,那么堆中第一个元素的值是最大值。区间为左闭右开区间。

\begin{itemize}
  \item make\_heap(begin, end, comp):将某区间内的元素转化为堆。时间复杂度$\mathcal{O}(n)$。
  \item push\_heap(begin, end, comp):假设[begin, end - 1)已经是一个堆。现在将end之前的那个元素加入堆中,使区间[begin, end)重新成为堆。时间复杂度$\mathcal{O}(\log n)$。
  \item pop\_heap(begin, end, comp):从区间[begin, end)取出第一个元素,放到最后位置,然后将区间[begin, end - 1)重新组成堆。时间复杂度$\mathcal{O}(\log n)$。
  \item sort\_heap(begin, end, comp):将heap转换为一个有序集合。时间复杂度$\mathcal{O}(n\log n)$。
\end{itemize}

\subsection{STL中的优先队列}

priority\_queue基于vector实现(模板接受三个参数,第二个就是容器类型。一般使用vector,也可使用deque)。普通的队列是先进先出,而优先队列是按照优先级出队,即无论入队顺序如何,出队的都是最大(最小)值。

priority\_queue位于<queue>,而greater和less存在于<functional>。

可以用以下几种方式定义优先队列(假设arr是一个有10个元素的数组):

\begin{lstlisting}
// 元素为int类型,最大值先出列。
priority_queue<int> q1;
// 元素为int类型,最小值先出列。 注意两个 > 之间有空格。
priority_queue<int, vector<int>, greater<int> > q2;
// 元素为float类型,最大值先出列,用现有数组初始化。
priority_queue<float> q3(arr, arr + 10);
\end{lstlisting}

优先队列支持的操作有:push、top(不是 front)、pop、empty和size。

如果需要使用自己的结构体,你需要重载复制构造函数和“>”或“<”运算符。 less对应“<”,表示最大值先出列;greater对应“>”,表示最小值先出列。

\begin{lstlisting}
struct MyStruct {
  int v;
  MyStruct(int i) : v(i) {}
  bool operator <(const MyStruct& b) const {
    return v < b.v;
  }
};
priority_queue<MyStruct, vector<MyStruct>, less<MyStruct> > q;
\end{lstlisting}


\section{二叉树}

\paragraph{例题} HDU1710(根据前序序列和中序序列构造二叉树)

\lstinputlisting{data_structure/binary_tree.cpp}

\section{KD树}

\paragraph{例题} HDU2966(寻找每个点距离最近点的距离),HDU4347(寻找距离某个点最近的m个点)

\lstinputlisting{data_structure/kd_tree.cpp}

\section{哈希表}
\lstinputlisting{data_structure/hash_table.cpp}
