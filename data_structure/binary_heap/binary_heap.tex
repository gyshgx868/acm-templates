\subsection{基本二叉堆}

二叉堆:二叉堆是完全二叉树。按照树根大小,二叉堆可分为最大值堆和最小值堆。

二叉堆的特点:

\begin{enumerate}
  \item 最大(小)值堆中,结点一定不小(大)于两个儿子的值;
  \item 在堆中,两兄弟的大小没有必然联系;
  \item 最大(小)值堆的根结点是整个树中的最大(小)值。
\end{enumerate}

由于是完全二叉树,所以可以直接用一维数组保存。数组的下标是从0开始的。

二叉堆的操作有:

\begin{enumerate}
  \item 插入:在堆中插入元素,首先要把元素放到末尾,然后通过不断往上“拱”,把元素“拱”到正确的位置;
  \item 用现有值初始化:最快的方法不是挨个插入,而是直接调整数组元素的顺序,使其符合堆的性质;
  \item 查找:查找最值是最快的——直接访问树根就可以了。不过,用堆查找其他值就很慢了。因此,可以考虑再使用一个适合查找的辅助数据结构,例如二叉排序树;
  \item 删除:把堆中最后一个元素 (就是一维数组存储所对应的最后一个元素) 放到待删除元素的位置,将元素总数减1,然后调整各元素的顺序。
\end{enumerate}

\lstinputlisting{data_structure/binary_heap/heap.cpp}

\subsection{STL中的堆算法}

为了实现堆算法,需要一个支持随机迭代器的容器。当然,一维数组也可以。

下面各函数的comp用于代替默认的小于号。如果不需要,可以省略。如果不指明,那么堆中第一个元素的值是最大值。区间为左闭右开区间。

\begin{itemize}
  \item make\_heap(begin, end, comp):将某区间内的元素转化为堆。时间复杂度$\mathcal{O}(n)$。
  \item push\_heap(begin, end, comp):假设[begin, end - 1)已经是一个堆。现在将end之前的那个元素加入堆中,使区间[begin, end)重新成为堆。时间复杂度$\mathcal{O}(\log n)$。
  \item pop\_heap(begin, end, comp):从区间[begin, end)取出第一个元素,放到最后位置,然后将区间[begin, end - 1)重新组成堆。时间复杂度$\mathcal{O}(\log n)$。
  \item sort\_heap(begin, end, comp):将heap转换为一个有序集合。时间复杂度$\mathcal{O}(n\log n)$。
\end{itemize}

\subsection{STL中的优先队列}

priority\_queue基于vector实现(模板接受三个参数,第二个就是容器类型。一般使用vector,也可使用deque)。普通的队列是先进先出,而优先队列是按照优先级出队,即无论入队顺序如何,出队的都是最大(最小)值。

priority\_queue位于<queue>,而greater和less存在于<functional>。

可以用以下几种方式定义优先队列(假设arr是一个有10个元素的数组):

\begin{lstlisting}
// 元素为int类型,最大值先出列。
priority_queue<int> q1;
// 元素为int类型,最小值先出列。 注意两个 > 之间有空格。
priority_queue<int, vector<int>, greater<int> > q2;
// 元素为float类型,最大值先出列,用现有数组初始化。
priority_queue<float> q3(arr, arr + 10);
\end{lstlisting}

优先队列支持的操作有:push、top(不是 front)、pop、empty和size。

如果需要使用自己的结构体,你需要重载复制构造函数和“>”或“<”运算符。 less对应“<”,表示最大值先出列;greater对应“>”,表示最小值先出列。

\begin{lstlisting}
struct MyStruct {
  int v;
  MyStruct(int i) : v(i) {}
  bool operator <(const MyStruct& b) const {
    return v < b.v;
  }
};
priority_queue<MyStruct, vector<MyStruct>, less<MyStruct> > q;
\end{lstlisting}
