\section{盲目搜索}

\subsection{纯随机搜索}

纯随机搜索适合情况深度很大,可行解很多且解的深度不重要的情况。尽管可以防止某些经过故意设计的数据,但是时间仍然是指数级的。

\subsection{深度优先搜索}

深度优先搜索是自顶向下地处理问题的方法,能够很快到达解答树的底端。

假设每次搜索产生的结点数为$N$,搜索的最大深度为$d$,那么时间复杂度为$\mathcal{O}(N^d)$。深度优先搜索比较费时,如果不剪枝,时间复杂度将是指数级的。如果状态空间不是有限的(可以无限扩展),那么深度优先搜索将无法得到解。此外,如果结点数太多,要注意可能会有爆栈的危险。

下面是深度优先搜索的一个框架。首先是很容易理解的递归版本:

\lstinputlisting{search/recursive_dfs.cpp}

下面是非递归版本。把广度优先搜索中的队列改成栈,就可以把广度优先搜索变成深度优先搜索。

\lstinputlisting{search/nonrecursive_dfs.cpp}

\subsection{广度优先搜索}

由于广度优先搜索是“扩散式”的搜索,所以解靠近解答树的根结点,也就是说,广度优先搜索适合解决某些“步数最少”、“深度最小”的问题。

深搜和广搜的不同之处就在于搜索顺序。假设每次搜索产生的结点数为$N$,搜索的最大深度为$d$,则广度优先搜索的时间开销为$\mathcal{O}(N^d)$。广度优先搜索速度比较快,但是很占空间,而且在扩展结点时容易产生重复结点。因此,一般情况下,在广搜扩展结点之前,需要用某些数据结构来判重。

\lstinputlisting{search/bfs.cpp}

\subsection{迭代加深搜索}

迭代加深搜索(Iterative Deepening Depth-First Search, IDS or IDDFS)是对状态空间的搜索策略。它重复地运行一个有深度限制的深度优先搜索,每次运行结束后,它增加深度并迭代,直到找到目标状态。具体操作是:限定深度优先搜索的深度,而且使深度从小到大递增。举个例子,就是先令最大深度为3,然后进行搜索。如果没有找到解,就令最大深度为4再重新搜素。

很明显,IDS能够搜索到所有节点,但是会进行很多重复的工作。IDS扩展结点数在渐近意义下和广度优先搜索是相同的,不过,\textbf{IDS空间复杂度和深度优先搜索差不多}。IDS综合了深度优先搜索和广度优先搜索的优势,是一种实用的盲目搜索算法。

\lstinputlisting{search/ids.cpp}

\section{双向广度优先搜索}

双向广度优先搜索的特点:从初始结点和目标结点开始分别做广度优先搜索,每次检测两边是否重合。每次扩展结点后,一般选择结点比较少的一侧继续搜索。假设每次搜索扩展$N$个结点,单纯使用广度优先搜索产生的深度为$d$,那么生成的结点数就为$\mathcal{O}(N^{d/2})$。

双向广度优先搜索的适用范围比较窄。使用该算法需要注意以下两个问题:

\begin{enumerate}
  \item 初始结点和目标结点是确定的,而且正向扩展结点和反向扩展结点的过程是易于相互转化的;
  \item 能够很快地检测两边搜索的状态是否重合。
\end{enumerate}

除了广度优先搜索,深度优先搜索和迭代加深搜索也可以被改造成双向搜索。

\lstinputlisting{search/bidirectional_bfs.cpp}

\section{二分查找}

二分查找要求数据单调有序,时间复杂度为$\mathcal{O}(\log n)$了。

\subsection{实现}
\lstinputlisting{search/binary_search.cpp}

\subsection{二分法求上下界}

\paragraph{求下界} 寻找$a$的区间$[x, y)$中大于等于$v$的第一个数,使得$v$插入到$a$的对应位置以后,$a$还是一个有序的数组。

\lstinputlisting{search/lower_bound.cpp}

\paragraph{求上界} 寻找$a$的区间$[x, y)$中大于$v$的第一个数,使得$v$插入到$a$对应位置的前面一位之后,$a$还是一个有序的数组。

\lstinputlisting{search/upper_bound.cpp}

\subsection{STL中的二分查找}

STL中有二分查找的算法,位于头文件<algorithm>中。

\begin{itemize}
  \item binary\_search(begin, end, value[, comp])\\判断已序区间[begin, end)内是否包含和value相等的元素。如果省略comp则使用“<”运算符进行比较。注意,返回值只说明是否存在,不指明位置。
  \item lower\_bound(begin, end, value[, comp])\\返回第一个大于等于value的元素地址,即可以插入value且不破坏区间有序性的第一个位置。省略comp则使用“<”比较。
  \item upper\_bound(begin, end, value[, comp])\\返回第一个大于value的元素地址,即可以插入value且不破坏区间有序性的最后一个位置。省略comp则使用“<”比较。
\end{itemize}

\subsection{二分答案}

许多问题不易直接求解,但我们可以用很短的时间来判断某个解是否可行,而且已知候选答案的范围$[\mathrm{min},\mathrm{max}]$。这时我们大可不必通过计算得到答案,只需在此范围内应用“二分”的过程,逐渐靠近答案。

当然不是任何题目都适合使用“二分答案”。能够二分答案的题目有以下几个特点:

\begin{itemize}
  \item 候选答案必须是离散的,且答案在一个上下界确定的范围里;
  \item 候选答案在区间$[\mathrm{min},\mathrm{max}]$上是有序的;
  \item 很容易去判断某个值是不是答案。
\end{itemize}

\section{三分查找}
\lstinputlisting{search/ternary_search.cpp}

\section{记录路径}

有的搜索问题是需要记录路径的。一般采用以下三种记录路径的方法:

\begin{enumerate}
  \item 在结点内部记录路径。在路径不太长的情况下是可行的;
  \item 记录“结点是从哪里来的”。搜索结束后,可以通过递归或迭代来找到整条路径;
  \item 维护一个链表,用链表记录路径。建议使用这种方法。
\end{enumerate}

\section{搜索的优化}

搜索浪费时间,主要体现在两个方面:一是产生了无用结点,二是产生了重复结点。对搜索算法进行优化,也就需要从这两方面入手。

\subsection{剪枝}

剪枝可以分为可行性剪枝和最优化剪枝。

所谓可行性剪枝,就是剪掉“明显不成立”、“不合理”的结点。在扩展结点之前,可以先判断这个结点是否合理,如果不合理则不扩展。而最优化剪枝用于最优化问题。在扩展结点之前,可以先估计一下结点的上下界。例如,对于一道找“最小值”的问题,我们肯定要记录已经搜索到最小值。如果扩展结点的时候发现结点的值已经大于这个最小值(或者此结点加上后续结点能取到的最小值的总和比这个最小值大),那么无论怎样搜索都不可能再得到最小值,这样就可以剪枝了。

\subsection{判重}

如果能够建立解答树的模型,或者建立出的隐式图是无环的,或者在搜索时有避免重复的结点扩展方式,那么我们不需要特意考虑判重的问题。

对于深度优先搜索来说,我们可以采用记忆化搜索的办法来避免重复搜索;对于广度优先搜索来说,判重是一个需要考虑的问题。一般情况下可以使用散列表来判断结点是否重复,如果发现某结点是重复的,那么就不再扩展该结点。

\subsection{预处理}

在付诸暴力之前,建议大家先找一找数据的规律,进行一些预处理。例如寻找规律、有序化处理等。如果有多个条件,则需要考虑先从哪个条件入手才能减少搜素量,而且需要考虑从哪个条件入手更有利于剪枝。

\subsection{Dancing Links}

\subsubsection{Dancing Links 精确覆盖}
\lstinputlisting{search/dancing_links_exact_cover.cpp}

\subsubsection{Dancing Links 重复覆盖}
\lstinputlisting{search/dancing_links_repeated_cover.cpp}
