\section{威佐夫博奕}

有两堆石子,石子数目分别为$n$和$m$,现在两个人轮流从两堆石子中拿石子,每次拿时可以从一堆石子中拿走若干个,也可以从两中拿走相同数量的石子,拿走最后一颗石子的人赢。

可以发现,前面几组奇异局势为$(a, b)$:$(0, 0)$、$(1, 2)$、$(3, 5)$、$(4, 7)$、$(6, 10)$、$(8, 13)$、$(9, 15)$、$(11, 18)$、$(12, 20)$

这其中第$k$组的$a为$前面没有出现过的最小非负整数,而$b=a+k$。两个人如果都采用正确操作,那么面对非奇异局势,先拿者必胜;反之,则后拿者取胜。

那么任给一个局势$(a, b)$,怎样判断它是不是奇异局势呢?我们有如下公式:

\begin{equation}
  a_k=\left\lfloor\frac{k(1+\sqrt{5})}{2}\right\rfloor, \; b_k=a_k+k, \; k=0,1,2, \dots , n
\end{equation}

\lstinputlisting{game_theory/wythoff.cpp}

\section{尼姆博弈}

\paragraph{定义} 有三堆各若干个物品,两个人轮流从某一堆取任意多的物品,规定每次至少取一个,多者不限,最后取光者得胜。

\paragraph{分析} 当剩下一堆石头时,一定是必胜态。当剩下两堆相同的石头时$(n, n)$,一定是必败态。只要你取$x$,对手在另一堆里面取$x$,会得到$(n – x, n – x)$。至于三堆的就比较麻烦,可以知道$(1, 2, 3)$是必败态。

\paragraph{结论} 当每堆石头取异或之后,不为0代表着必胜态,为0代表着必败态。

\lstinputlisting{game_theory/nim.cpp}

\section{巴什博奕}

\paragraph{问题} 有一堆石子,石子个数为$n$,两人轮流从石子中取石子出来,最少取一个,最多取$m$个。最后不能取的人输,问你最后的输赢情况。

\paragraph{思路} 这种就是可以直接判断必败态的问题。如果这堆石子少于或者等于$m$个,那么先手赢。如果石子数目为$m+1$个,那么先手必输,因为无论先手怎么拿石子,后手都可以直接把剩下的石子全部拿走。如果石子数目为$m+1<n<2(m + 1)$,那么先手就可以拿走$n-(m+1)$个石子,使得对手面对$m+1$的必败态,这样先手必赢。所以如此推算下去,我们就知道当$n \mod (m+1)=0$时,先手输,否则先手赢。

\lstinputlisting{game_theory/bash.cpp}
