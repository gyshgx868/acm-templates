\section{C++中的字符串}

\subsection{string类}

有些字符串操作本来很简单,但是在C语言里就很麻烦。C++通过std::string类,在一定程度上缓解了这个问题,但是速度很慢。std::string的头文件为<string>。

cin、cout可以直接输入、输出string字符串。如果需要整行读入,可以用getline函数:

\begin{lstlisting}
string st;
cin >> st;
while (getline(cin, st)) {
  cout << st << endl;
}
\end{lstlisting}

基本操作:string类相当于vector<char>,所以string可以使用vector所支持的大部分操作,例如:

\begin{lstlisting}
string st = "abcde";
cout << st[1] << endl;
st.push_back('f');
\end{lstlisting}

\begin{itemize}
  \item 可以用“+”连接两个字符串。当然,两个字符串中需要至少一个是string类型,否则会编译错误。由于涉及内存分配,不建议使用;
  \item 用==、!=判断两个字符串是否相等。用比较符号比较两个字符串的大小;
  \item 用+=运算符或st.append()在st后面附加字符串;
  \item 用st.size()获得字符串st的长度;
  \item 用st.find()在字符串里寻找子串;如果没找到,函数会返回string::npos; \\
        size\_t p = st.find("hello", 0); // 第二个参数表示起始位置
  \item 用st.substr(5, 10)来截取从第5个字符开始、长度为10的字符串。如果长度为string::npos,则一直截取到字符串结束;
  \item 用st.c\_str()获取一个字符串数组,这样就可以继续用C语言的处理方式来处理字符串了。
\end{itemize}

\subsection{stringstream}

stringstream可用于从字符串中读取内容,头文件为<sstream>。注意,stringstream的速度非常慢。

以下是“输入若干行整数,求它们的和”的代码:

\begin{lstlisting}
string st;
while (getline(cin, st)) {
  stringstream ss(st);
  int sum = 0, p;
  while (ss >> p) { sum += p; }
  cout << sum << endl;
}
\end{lstlisting}

\section{字符串匹配}

\paragraph{问题} 已知原串和模式串,求模式串在原串中的个数和位置。

对于此类问题,很容易想到朴素算法,或者是strstr()与string的find()。不过,在字符串很长的情况下,朴素算法会浪费时间。C和C++标准也没有对strstr和find的实现方式做出规定,而且我们碰到的问题也并不总是单纯的字符串,所以我们需要了解更高效的算法。

常用的串匹配算法有KMP算法、BM算法、KR算法等。

\subsection{KMP算法}

KMP算法的原理:假设原串S为abababacabcababacbc,模式串P为ababacb。

按照朴素算法的做法,发现不匹配之后,应该把模式串向右移动一位。而KMP算法会根据预处理的结果来决定应该把模式串向右移动多少位。

\lstinputlisting{string/kmp.cpp}

\subsection{扩展KMP (Z函数)}
\lstinputlisting{string/z_algorithm.cpp}

\section{字符串哈希}

\paragraph{例题} POJ1200(给定字符串以及一个长度$n$,询问该字符串中长度为$n$的不同子串有多少个)

\lstinputlisting{string/string_hash.cpp}

\section{字典树}

query函数返回以s为前缀的单词共有多少个。

\paragraph{例题} HDU1251

\lstinputlisting{string/trie.cpp}

\section{AC自动机}

\paragraph{例题} HDU2222

\lstinputlisting{string/aho_corasick.cpp}

\section{后缀数组}

\paragraph{例题} 洛谷P2408(不同子串个数)

\lstinputlisting{string/suffix_array.cpp}

\section{回文树}

\paragraph{例题} HDU5421,洛谷P5496(回文自动机)

\lstinputlisting{string/palindromic_tree.cpp}

\section{字符串表示法}

\subsection{最小表示法}

对于一个字符串$S$,求$S$的循环的同构字符串$S^{'}$中字典序最小的一个。例如,序列
\[
  [10,9,8,7,6,5,4,3,2,1]
\]
的最小表示为
\[
  [1,10,9,8,7,6,5,4,3,2]
\]

\paragraph{例题} 洛谷P1368

\lstinputlisting{string/min_representation.cpp}

\subsection{最大表示法}
\lstinputlisting{string/max_representation.cpp}

\section{最长回文字串}
\lstinputlisting{string/manacher.cpp}
